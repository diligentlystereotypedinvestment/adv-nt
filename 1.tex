\documentclass[11pt]{article}
\usepackage{hyperref}
\usepackage[]{microtype}
\usepackage{amsthm}
\usepackage{amsmath, amssymb}
\theoremstyle{definition}
\newtheorem{theorem}{{Theorem}}
\newtheorem{lemma}{{Lemma}}
\newtheorem{proposition}{{Proposition}}
\newtheorem{definition}{{Definition}}
\newtheorem{corollary}{{Corollary}}
\newtheorem{example}{{Example}}
\newtheorem{conjecture}{{Conjecture}}
\newtheorem{remark}{{Remark}}
\newtheorem{claim}{{Claim}}
\newtheorem{exercise}{{Exercise}}
\usepackage{biblatex} %Imports biblatex package
\addbibresource{bib.bib} %Import the bibliography file
\begin{document}
\author{Vincent Tran}
\title{Proofs of the Exercises in Chapter 1}
\maketitle

\begin{exercise}
	\begin{proof}
	Let \(a,b,c,d,e,f\in\mathbb{Z} \) and \(x,y,z\in\mathbb{Z}/m \) such that \((b,m) = (d,m) = (f,m) = 1 \) and \(xb  \equiv a, yd\equiv c, zf \equiv e \pmod{m}\).
	Then we can show that \(\frac{a}{b}+\frac{c}{d}=\frac{e}{f} \implies  x+y\equiv z\):
	\begin{align*}
		\frac{a}{b}+\frac{c}{d}=\frac{e}{f}\\
		adf + cbf = ebd\\
		xbdf + ybdf \equiv zbdf\\
		x+y\equiv z
	\end{align*}
	Furthermore, if \(\frac{ac}{bd}= \frac{e}{f}\) for possibly different \(e,f \), then \(xy \equiv z \) for possible different \(z \):
	\begin{align*}
		\frac{ac}{bd}=\frac{e}{f}\\
		acf = bed\\
		xbydf \equiv bzfd\\
		xy \equiv z
	\end{align*}
	\end{proof}
\end{exercise}

\begin{exercise}
	\begin{proof}
	Consider the sequence \(\frac{1}{1},\frac{1}{2},\frac{1}{3}, \cdots \frac{1}{p-1} \). We can show that no two elements in this series are equivalent mod odd \(p \). FTOC suppose there are two integers \(a,b \) such that \(\frac{1}{a} \equiv\frac{1}{b}\), let this be equal to \(x \). Then \(ax \equiv 1 \equiv bx \). Since \(p \) is prime, all integers in \([1,p-1] \) are relatively prime to \(p \). Thus \(ax\equiv bx \implies a\equiv b \), which is impossible since \(a\ne b \)  and \(a + pk > p-1 \) for \(k\in\mathbb{N} \) as \(a \ge 1 \). Thus every fraction is uniquely equivalent to an element in \(\mathbb{Z}/p \).

	Thus the sum of all the elements in this series is just \(1+2+3+4+\cdots +p-1 \) as every element of \(\mathbb{Z}/p \) must be in this series (there are \(p-1 \) elements in \(\mathbb{Z}/p \) and \(p-1 \) unique elements in the sequence). Thus the sum is just \((p-1)(p)/2 \equiv 0\).
	\end{proof}
\end{exercise}

\begin{exercise}
	\begin{proof}
		By the hypothesis, \(x^4 + y^4 = x^7y + 1 \equiv 0 \pmod{x^4+y^4}\). Thus \(y \equiv \frac{-1}{x^7}\implies x^4 + (\frac{-1}{x^7})^4 = \frac{x^{32} + 1}{x^{28}} \equiv 0 \implies x^{32} + 1 \equiv 0\).
	\end{proof}
\end{exercise}

\begin{exercise}
	\begin{proof}

	\end{proof}
\end{exercise}

\end{document}
