\documentclass[11pt]{article}
\usepackage{hyperref}
\usepackage[]{microtype}
\usepackage{amsthm}
\usepackage{amsmath, amssymb}
\theoremstyle{definition}
\newtheorem{theorem}{{Theorem}}
\newtheorem{lemma}{{Lemma}}
\newtheorem{proposition}{{Proposition}}
\newtheorem{definition}{{Definition}}
\newtheorem{corollary}{{Corollary}}
\newtheorem{example}{{Example}}
\newtheorem{conjecture}{{Conjecture}}
\newtheorem{remark}{{Remark}}
\newtheorem{claim}{{Claim}}
\newtheorem{exercise}{{Exercise}}
\usepackage{biblatex} %Imports biblatex package
\addbibresource{bib.bib} %Import the bibliography file
\newcommand{\ind}{\text{ind }}

\begin{document}
\author{Vincent Tran}
\title{Proofs of the Exercises in Chapter 1}
\maketitle

\begin{exercise}
	\begin{proof}
	Let \(a,b,c,d,e,f\in\mathbb{Z} \) and \(x,y,z\in\mathbb{Z}/m \) such that \((b,m) = (d,m) = (f,m) = 1 \) and \(xb  \equiv a, yd\equiv c, zf \equiv e \pmod{m}\).
	Then we can show that \(\frac{a}{b}+\frac{c}{d}=\frac{e}{f} \implies  x+y\equiv z\):
	\begin{align*}
		\frac{a}{b}+\frac{c}{d}=\frac{e}{f}\\
		adf + cbf = ebd\\
		xbdf + ybdf \equiv zbdf\\
		x+y\equiv z
	\end{align*}
	Furthermore, if \(\frac{ac}{bd}= \frac{e}{f}\) for possibly different \(e,f \), then \(xy \equiv z \) for possible different \(z \):
	\begin{align*}
		\frac{ac}{bd}=\frac{e}{f}\\
		acf = bed\\
		xbydf \equiv bzfd\\
		xy \equiv z
	\end{align*}
	\end{proof}
\end{exercise}

\begin{exercise}
	\begin{proof}
	Consider the sequence \(\frac{1}{1},\frac{1}{2},\frac{1}{3}, \cdots \frac{1}{p-1} \). We can show that no two elements in this series are equivalent mod odd \(p \). FTOC suppose there are two integers \(a,b \) such that \(\frac{1}{a} \equiv\frac{1}{b}\), let this be equal to \(x \). Then \(ax \equiv 1 \equiv bx \). Since \(p \) is prime, all integers in \([1,p-1] \) are relatively prime to \(p \). Thus \(ax\equiv bx \implies a\equiv b \), which is impossible since \(a\ne b \)  and \(a + pk > p-1 \) for \(k\in\mathbb{N} \) as \(a \ge 1 \). Thus every fraction is uniquely equivalent to an element in \(\mathbb{Z}/p \).

	Thus the sum of all the elements in this series is just \(1+2+3+4+\cdots +p-1 \) as every element of \(\mathbb{Z}/p \) must be in this series (there are \(p-1 \) elements in \(\mathbb{Z}/p \) and \(p-1 \) unique elements in the sequence). Thus the sum is just \((p-1)(p)/2 \equiv 0\).
	\end{proof}
\end{exercise}

\begin{exercise}
	\begin{proof}
		By the hypothesis, \(x^4 + y^4 = x^7y + 1 \equiv 0 \pmod{x^4+y^4}\). Thus \(y \equiv \frac{-1}{x^7}\implies x^4 + (\frac{-1}{x^7})^4 = \frac{x^{32} + 1}{x^{28}} \equiv 0 \implies x^{32} + 1 \equiv 0\).
	\end{proof}
\end{exercise}

\begin{exercise}
	\begin{proof}
		Let \((a,m)=d \). Thus \(a=dk \) for some \(k\in\mathbb{Z} \) such that \((k,m) = 1 \). Thus we have it that \(dkx \equiv b \). Thus \(dkx = b + mj \) for some \(j\in\mathbb{Z} \). Since \(d\mid m \), \(d\mid b \). Then we can show that if \(d\mid b, \exists x \). We can show that \(x\equiv k^{-1}(e+fj) \) where \(ed=b \) and \(fd=m \) is a solution: \(ax \equiv dkk^{-1}(e+fj) \equiv de+dfj \equiv b + mj \equiv b \). Thus \(\exists x \) iff \(d|b \).
	\end{proof}
\end{exercise}

\begin{exercise}
	\begin{proof}
		Let \(x = 3a + r_1 = 4b + r_2 = 5c + r_3\) for some \(a,b,c\in\mathbb{Z} \). Then \(3a + r_1 \equiv r_2 \pmod{4} \). Thus \(3a \equiv r_2 - r_1 \implies a \equiv 3r_2 - 3r_1 \equiv 3r_2 + r_1 \). Therefore \(a = 3r_2 + r_1 + 4d \) for some integer \(d \). Substituting into \(x \), we get \(x = 9r_2 + 3r_1 + 12d \). We can repeat this with \(x =  5c + r_3\) to get the result.
	\end{proof}
\end{exercise}

\begin{exercise}
	\begin{proof}
	The verify part is just back-of-the-napkin math, so I'll leave it out of this. Then we do some algebra:
	\begin{align*}
		10\ind 2 + 60\ind y \equiv 70\ind 14 \pmod{19}\\
		1 + 6\ind y \equiv 7 \cdot 7\\
		6\ind y \equiv 48\\
		\ind y \equiv 8\\
		y \equiv 9
	\end{align*}
	\end{proof}
\end{exercise}

\begin{exercise}
	\begin{proof}
		\[
			\ind y = [t_0, t_1, t_2] \implies \ind y^2 = [2t_0, 2t_1, 2t_2]
		\]
		Thus \(y^2 \equiv 2^{2t_2} \equiv 1 \pmod{3} \) and \(y^2 \equiv (-1)^{2t_0}5^{2t_1}  \equiv 1 \pmod{8}\). Thus \(y^2 \equiv 1 \).
	\end{proof}
\end{exercise}

\begin{exercise}
	\begin{proof}
		Let \(\ind y = [t_0, t_1, t_2, \cdots , t_s] \). Thus \(\ind y^4 = [4t_0, \cdots, 4t_s] \). For a non-trivial solution, \(4t_i \equiv 0 \pmod{\phi(p_i^{a_i})} \). Since \(t_i \) varies, the only way for this to occur is for \(\phi(p_i^{a_i})|4 \implies \phi(p_i^{a_i}) = 1,2,4 \). This is only the case for \(p_i^{a_i}=3,5,4,8,16 \).
	\end{proof}
\end{exercise}

\begin{exercise}
	\begin{proof}
		Suppose there is a \(a_i\otimes a_s \in K_i \) and \(a_j\otimes a_r \in K_j \) such that \(a_i \otimes a_s = a_j\otimes a_r \). Then \(a_i = a_j \otimes a_r \otimes a_s^{-1} \). Thus \(K_i = \{a_j \otimes a_r \otimes a_s^{-1} \otimes a_1 \cdots  \). Since \(a_r, a_s, a_n \in K\) for \(n \in [1, t]\), \(a_r \otimes a_s^{-1} \otimes a_n \in K \). It is trivial to see that for all  \(n \), this product is unique. Thus every element of \(K_i \) is a unique product of \(a_j \) and an element of \(K \), thus making \(K_i =K_j \) if they share an element. Thus if they aren't equal, they share no elements.

		Since \(\exists a_x \in G \forall a_n \in K: a_x \otimes a_n = a_y\) for \(a_y \in G \), every element of \(G \) is in some \(K_x \). Since the \(K_x \)'s  have a fixed size and are disjoint, \(t\mid h \) since otherwise, there would be a remainder to \(h/t \) that are elements not in a \(K_x \), which is impossible. Since they are disjoint, have size \(t \), and make up the entire set of \(G \), \(\frac{h}{t} \) must be the number of cosets there are.
	\end{proof}
\end{exercise}

\end{document}
